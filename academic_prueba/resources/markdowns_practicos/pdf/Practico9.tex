\PassOptionsToPackage{unicode=true}{hyperref} % options for packages loaded elsewhere
\PassOptionsToPackage{hyphens}{url}
%
\documentclass[]{article}
\usepackage{lmodern}
\usepackage{amssymb,amsmath}
\usepackage{ifxetex,ifluatex}
\usepackage{fixltx2e} % provides \textsubscript
\ifnum 0\ifxetex 1\fi\ifluatex 1\fi=0 % if pdftex
  \usepackage[T1]{fontenc}
  \usepackage[utf8]{inputenc}
  \usepackage{textcomp} % provides euro and other symbols
\else % if luatex or xelatex
  \usepackage{unicode-math}
  \defaultfontfeatures{Ligatures=TeX,Scale=MatchLowercase}
\fi
% use upquote if available, for straight quotes in verbatim environments
\IfFileExists{upquote.sty}{\usepackage{upquote}}{}
% use microtype if available
\IfFileExists{microtype.sty}{%
\usepackage[]{microtype}
\UseMicrotypeSet[protrusion]{basicmath} % disable protrusion for tt fonts
}{}
\IfFileExists{parskip.sty}{%
\usepackage{parskip}
}{% else
\setlength{\parindent}{0pt}
\setlength{\parskip}{6pt plus 2pt minus 1pt}
}
\usepackage{hyperref}
\hypersetup{
            pdftitle={Práctico 9},
            pdfauthor={Funciones y Loops en R},
            pdfborder={0 0 0},
            breaklinks=true}
\urlstyle{same}  % don't use monospace font for urls
\usepackage[margin=1in]{geometry}
\usepackage{color}
\usepackage{fancyvrb}
\newcommand{\VerbBar}{|}
\newcommand{\VERB}{\Verb[commandchars=\\\{\}]}
\DefineVerbatimEnvironment{Highlighting}{Verbatim}{commandchars=\\\{\}}
% Add ',fontsize=\small' for more characters per line
\usepackage{framed}
\definecolor{shadecolor}{RGB}{248,248,248}
\newenvironment{Shaded}{\begin{snugshade}}{\end{snugshade}}
\newcommand{\AlertTok}[1]{\textcolor[rgb]{0.94,0.16,0.16}{#1}}
\newcommand{\AnnotationTok}[1]{\textcolor[rgb]{0.56,0.35,0.01}{\textbf{\textit{#1}}}}
\newcommand{\AttributeTok}[1]{\textcolor[rgb]{0.77,0.63,0.00}{#1}}
\newcommand{\BaseNTok}[1]{\textcolor[rgb]{0.00,0.00,0.81}{#1}}
\newcommand{\BuiltInTok}[1]{#1}
\newcommand{\CharTok}[1]{\textcolor[rgb]{0.31,0.60,0.02}{#1}}
\newcommand{\CommentTok}[1]{\textcolor[rgb]{0.56,0.35,0.01}{\textit{#1}}}
\newcommand{\CommentVarTok}[1]{\textcolor[rgb]{0.56,0.35,0.01}{\textbf{\textit{#1}}}}
\newcommand{\ConstantTok}[1]{\textcolor[rgb]{0.00,0.00,0.00}{#1}}
\newcommand{\ControlFlowTok}[1]{\textcolor[rgb]{0.13,0.29,0.53}{\textbf{#1}}}
\newcommand{\DataTypeTok}[1]{\textcolor[rgb]{0.13,0.29,0.53}{#1}}
\newcommand{\DecValTok}[1]{\textcolor[rgb]{0.00,0.00,0.81}{#1}}
\newcommand{\DocumentationTok}[1]{\textcolor[rgb]{0.56,0.35,0.01}{\textbf{\textit{#1}}}}
\newcommand{\ErrorTok}[1]{\textcolor[rgb]{0.64,0.00,0.00}{\textbf{#1}}}
\newcommand{\ExtensionTok}[1]{#1}
\newcommand{\FloatTok}[1]{\textcolor[rgb]{0.00,0.00,0.81}{#1}}
\newcommand{\FunctionTok}[1]{\textcolor[rgb]{0.00,0.00,0.00}{#1}}
\newcommand{\ImportTok}[1]{#1}
\newcommand{\InformationTok}[1]{\textcolor[rgb]{0.56,0.35,0.01}{\textbf{\textit{#1}}}}
\newcommand{\KeywordTok}[1]{\textcolor[rgb]{0.13,0.29,0.53}{\textbf{#1}}}
\newcommand{\NormalTok}[1]{#1}
\newcommand{\OperatorTok}[1]{\textcolor[rgb]{0.81,0.36,0.00}{\textbf{#1}}}
\newcommand{\OtherTok}[1]{\textcolor[rgb]{0.56,0.35,0.01}{#1}}
\newcommand{\PreprocessorTok}[1]{\textcolor[rgb]{0.56,0.35,0.01}{\textit{#1}}}
\newcommand{\RegionMarkerTok}[1]{#1}
\newcommand{\SpecialCharTok}[1]{\textcolor[rgb]{0.00,0.00,0.00}{#1}}
\newcommand{\SpecialStringTok}[1]{\textcolor[rgb]{0.31,0.60,0.02}{#1}}
\newcommand{\StringTok}[1]{\textcolor[rgb]{0.31,0.60,0.02}{#1}}
\newcommand{\VariableTok}[1]{\textcolor[rgb]{0.00,0.00,0.00}{#1}}
\newcommand{\VerbatimStringTok}[1]{\textcolor[rgb]{0.31,0.60,0.02}{#1}}
\newcommand{\WarningTok}[1]{\textcolor[rgb]{0.56,0.35,0.01}{\textbf{\textit{#1}}}}
\usepackage{graphicx,grffile}
\makeatletter
\def\maxwidth{\ifdim\Gin@nat@width>\linewidth\linewidth\else\Gin@nat@width\fi}
\def\maxheight{\ifdim\Gin@nat@height>\textheight\textheight\else\Gin@nat@height\fi}
\makeatother
% Scale images if necessary, so that they will not overflow the page
% margins by default, and it is still possible to overwrite the defaults
% using explicit options in \includegraphics[width, height, ...]{}
\setkeys{Gin}{width=\maxwidth,height=\maxheight,keepaspectratio}
\setlength{\emergencystretch}{3em}  % prevent overfull lines
\providecommand{\tightlist}{%
  \setlength{\itemsep}{0pt}\setlength{\parskip}{0pt}}
\setcounter{secnumdepth}{0}
% Redefines (sub)paragraphs to behave more like sections
\ifx\paragraph\undefined\else
\let\oldparagraph\paragraph
\renewcommand{\paragraph}[1]{\oldparagraph{#1}\mbox{}}
\fi
\ifx\subparagraph\undefined\else
\let\oldsubparagraph\subparagraph
\renewcommand{\subparagraph}[1]{\oldsubparagraph{#1}\mbox{}}
\fi

% set default figure placement to htbp
\makeatletter
\def\fps@figure{htbp}
\makeatother


\title{Práctico 9}
\author{Funciones y Loops en R}
\date{}

\begin{document}
\maketitle

\hypertarget{funciones}{%
\section{Funciones}\label{funciones}}

\textbf{Definicion de funciones en R}

El lenguaje R hace uso de múltiples funciones: las mismas se encuentran
pre-instaladas por defecto, o se pueden descargar de librerías
específicas programadas para el trabajo en diversas áreas. En el
práctico de la semana pasada, por ejemplo, se trabajó con funciones de
la librería seqinr, las cuales facilitan el trabajo con secuencias
biológicas en este lenguaje.

El usuario de R puede a su vez definir sus propias funciones. Esto es de
gran utilidad: si se define una función que realiza una función dada, no
es necesario reescribir el código que lleva a cabo este conjunto de
acciones cada vez que se quiere realizar las mismas sobre diferentes
variables.

Una función queda definida en R por dos elementos principales: el
conjunto de elementos sobre los que opera (sus \textbf{argumentos}), y
el \textbf{código} en el cual se especifica el conjunto de acciones que
realiza dicha función. Al igual que en el caso de loops, el código de
una función queda delimitado a través del uso de llaves.

A continuación se muestra un código simple para ilustrar estos
conceptos, donde se define una función sencilla. La misma toma un número
y devuelve el resultado de multiplicar el mismo por cuatro.

\begin{Shaded}
\begin{Highlighting}[]
    \CommentTok{# Se define la funcion multiplica_por_cuatro, la cual tiene como argumento un numero.}
\NormalTok{    multiplica_por_cuatro =}\StringTok{ }\ControlFlowTok{function}\NormalTok{(numero)\{}
    \CommentTok{# se guarda el resultado de multiplicar a la variable numero por 4}
\NormalTok{    resultado <-}\StringTok{ }\NormalTok{numero }\OperatorTok{*}\StringTok{ }\DecValTok{4}
    \CommentTok{# se devuelve al usuario el resultado}
    \KeywordTok{return}\NormalTok{(resultado)}
\NormalTok{    \}}
    
    \CommentTok{# se utiliza la funcion con diferentes numeros, y se guardan los resultados en }
    \CommentTok{# variables}
\NormalTok{    resultado_}\DecValTok{1}\NormalTok{ =}\StringTok{ }\KeywordTok{multiplica_por_cuatro}\NormalTok{(}\DataTypeTok{numero =} \DecValTok{1}\NormalTok{)}
\NormalTok{    resultado_}\DecValTok{3}\NormalTok{ =}\StringTok{ }\KeywordTok{multiplica_por_cuatro}\NormalTok{(}\DataTypeTok{numero =} \DecValTok{3}\NormalTok{)}
    
\NormalTok{    resultado_}\DecValTok{1}
\end{Highlighting}
\end{Shaded}

\begin{verbatim}
## [1] 4
\end{verbatim}

\begin{Shaded}
\begin{Highlighting}[]
\NormalTok{    resultado_}\DecValTok{3}
\end{Highlighting}
\end{Shaded}

\begin{verbatim}
## [1] 12
\end{verbatim}

Algo a destacar es que los argumentos de una función son representados
por variables (en este caso la variable \emph{numero}, las cuales son
empleadas para representar a estos argumentos en el código de la
función.

En el caso anterior la funcion \emph{multiplica\_por\_cuatro} siempre
devuelve el resultado de multiplicar el argumento \emph{numero} por
cuatro. A continuación se muestra una función más flexible, donde el
usuario puede pasar como argumento el número por el cual quiere
multiplicar a otro número

\begin{Shaded}
\begin{Highlighting}[]
    \CommentTok{# se define una nueva funcion, que toma como argumentos dos numeros y los multiplica.}
\NormalTok{    multiplica =}\StringTok{ }\ControlFlowTok{function}\NormalTok{(numero_}\DecValTok{1}\NormalTok{, numero_}\DecValTok{2}\NormalTok{)\{}
    \CommentTok{# se guarda el resultado de multiplicar ambos numeros}
\NormalTok{    resultado <-}\StringTok{ }\NormalTok{numero_}\DecValTok{1} \OperatorTok{*}\StringTok{ }\NormalTok{numero_}\DecValTok{2}
    \CommentTok{# se devuelve al usuario el resultado}
    \KeywordTok{return}\NormalTok{(resultado)}
\NormalTok{    \}}
    
    \CommentTok{# se utiliza la funcion con diferentes numeros, y se guardan los resultados en }
    \CommentTok{# variables}
\NormalTok{    resultado_}\DecValTok{1}\NormalTok{ =}\StringTok{ }\KeywordTok{multiplica}\NormalTok{(}\DataTypeTok{numero_1 =} \DecValTok{2}\NormalTok{, }\DataTypeTok{numero_2 =} \DecValTok{5}\NormalTok{)}
\NormalTok{    resultado_}\DecValTok{2}\NormalTok{ =}\StringTok{ }\KeywordTok{multiplica}\NormalTok{(}\DataTypeTok{numero_1 =} \DecValTok{3}\NormalTok{, }\DataTypeTok{numero_2 =} \DecValTok{17}\NormalTok{)}
    
\NormalTok{    resultado_}\DecValTok{1}
\end{Highlighting}
\end{Shaded}

\begin{verbatim}
## [1] 10
\end{verbatim}

\begin{Shaded}
\begin{Highlighting}[]
\NormalTok{    resultado_}\DecValTok{2}
\end{Highlighting}
\end{Shaded}

\begin{verbatim}
## [1] 51
\end{verbatim}

\hypertarget{ejercicio-1}{%
\subsection{Ejercicio 1}\label{ejercicio-1}}

Defina funciones que realicen las siguientes tareas:

\begin{quote}
\emph{a)} Tome como argumento a un nombre e imprima a pantalla un saludo
\end{quote}

\begin{quote}
\emph{b)} Dado un número, compute su raíz cuadrada.
\end{quote}

\begin{quote}
\textbf{Nota}: En caso de que el argumento no sea numerico, devuelva al
usuario un mensaje.
\end{quote}

\begin{quote}
\emph{c)} Dado un número, compute su raíz cuadrada. Si el número no es
entero, redondee primero y luego devuelva la raíz
\end{quote}

\begin{quote}
\textbf{Nota}: puede usar la función round().
\end{quote}

\begin{quote}
\emph{d)} Dados dos vectores, uno con nombres y otro con números de
teléfono, se devuelva un data.frame con los mismos. El mismo deberá
indexar a dichas personas con números (ver ejemplo a continuación).
Considere, ademas, que se realice la operacion si poseen igual largo los
vectores.
\end{quote}

\hypertarget{ejercicio-2}{%
\subsection{Ejercicio 2}\label{ejercicio-2}}

\begin{quote}
\emph{a)} Defina una función que, dado un conjunto de números, devuelva
el mayor de los mismos.
\end{quote}

\begin{quote}
\emph{b)} Cree una función que, dada una palabra, de la posición de sus
letras en un diccionario. La palabra \emph{``vaca''} da como resultado,
entonces, el vector {[}22, 1, 3, 1{]}
\end{quote}

\begin{quote}
\textbf{Nota:} utilice el vector \textbf{letters} (que se encuentra por
defecto en R) y la funcion s2c() de la libreria seqinr.
\end{quote}

\begin{quote}
\emph{c)} Defina una función que, dada una palabra, la devuelva en
mayúscula. Llame a la función creada en b) para esto, y el vector
LETTERS.
\end{quote}

\hypertarget{loops}{%
\section{Loops}\label{loops}}

La estructura de un \emph{for loop} en R es similar a la encontrada en
otros lenguajes (como Bash): \emph{i)} se define un conjunto de
elementos sobre los cuales se van a realizar acciones, \emph{ii)} se
define una variable con la cual se hace referencia a estos elementos y
\emph{iii)} se escriben las acciones a realizarse en el loop, las cuales
están delimitadas en un bloque.

A continuación se muestra un for loop sencillo. En el mismo se toma un
conjunto de números (\emph{i.e.} los números del 1 al 5), se les suma un
número y se guarda el resultado en un vector previamente definido.

\begin{Shaded}
\begin{Highlighting}[]
\CommentTok{# Se define el vector en el que se depositaran los resultados}
\NormalTok{resultados <-}\StringTok{ }\KeywordTok{c}\NormalTok{()}

\CommentTok{# Se declara que se realizara el for loop sobre los elementos del vector 1:5 }
\CommentTok{#(numeros del 1 al 5).}

\CommentTok{# A su vez se define la variable i, que representara a estos numeros }
\CommentTok{# en el loop}
\ControlFlowTok{for}\NormalTok{ (i }\ControlFlowTok{in} \DecValTok{1}\OperatorTok{:}\DecValTok{5}\NormalTok{) \{}
    \CommentTok{# Al numero i se le suma 3, y se deposita el resultado en el }
    \CommentTok{# i-esimo elemento del vector resultados}
\NormalTok{    resultados[i] <-}\StringTok{ }\NormalTok{i }\OperatorTok{+}\StringTok{ }\DecValTok{3}
\NormalTok{\}}
\end{Highlighting}
\end{Shaded}

En este caso, al comenzar el loop la variable \emph{i} tomará en primer
lugar el valor de 1. El resultado de sumar a este número 3 (lo cual se
realiza en el bloque de acciones a realizarse en el loop) es guardado en
el i-esimo elemento del vector \emph{resultados} (es decir, su primer
elemento).

Algo a destacar es el uso de las llaves (\emph{i.e.} \textbf{\{\}} para
delimitar el loop: estos cumplen un rol análogo al que cumplen las
palabras \textbf{do} y \textbf{done} en el lenguaje Bash, ayudando a
delimitar el bloque de acciones que se ejecutará durante el loop.

Otra forma de escribir el loop descrito arriba sería la siguiente:

\begin{Shaded}
\begin{Highlighting}[]
\CommentTok{# Se define el vector en el que se depositaran los resultados}
\NormalTok{resultados <-}\StringTok{ }\KeywordTok{c}\NormalTok{()}
\CommentTok{# Se define el vector de numeros del 1 al 5}
\NormalTok{numeros <-}\StringTok{ }\DecValTok{1}\OperatorTok{:}\DecValTok{5}
\CommentTok{# Se declara que se realizara el for loop sobre los elementos }
\CommentTok{# del vector numeros.}
\CommentTok{# A su vez se define la variable i, que representara a estos numeros }
\CommentTok{# en el loop}
\ControlFlowTok{for}\NormalTok{ (i }\ControlFlowTok{in} \DecValTok{1}\OperatorTok{:}\KeywordTok{length}\NormalTok{(numeros)) \{}
    \CommentTok{# Al numero i se le suma 3, y se deposita el resultado en el }
    \CommentTok{# i-esimo elemento del vector resultados}
\NormalTok{    resultados[i] <-}\StringTok{ }\NormalTok{numeros[i] }\OperatorTok{+}\StringTok{ }\DecValTok{3}
\NormalTok{\}}
\end{Highlighting}
\end{Shaded}

En esta versión no se hace referencia directa a los números, si no que
se realizan operaciones sobre los mismos a partir de los índices que
tienen en el vector \emph{numeros} (definido de forma previa al loop).
La variable \emph{i} toma, entonces, los valores desde 1 a
length(numeros) (el largo del vector numeros), recorriendo asi los
índices de los elementos que componen el vector. Estos índices son
usados para llamar a los elementos del mismo, lo cual se ve en el uso de
la expresión numeros{[}i{]}.

\hypertarget{ejercicio-1-1}{%
\subsection{Ejercicio 1}\label{ejercicio-1-1}}

\begin{quote}
\emph{a)} Elija alguna de las funciones que definió en la primer sección
del práctico. Aplique la misma sobre un conjunto de valores.
\end{quote}

\begin{quote}
\emph{b)} Modifique este loop para almacenar los resultados en otro
vector.
\end{quote}

\begin{quote}
\emph{c)} Realice la misma operación, pero valiendose de una función del
tipo *apply().
\end{quote}

\hypertarget{ejercicio-2-1}{%
\subsection{Ejercicio 2}\label{ejercicio-2-1}}

En el archivo \textbf{pedidoMesaUno.lista} se encuentra una tabla que
muestra el pedido que hizo un conjunto de personas en un bar
montevideano.

Los precios de los productos se encuentran en la tabla alojada en el
archivo \textbf{carta.lista}.

\begin{quote}
Utilizando un loop, obtenga:
\end{quote}

\begin{quote}
\emph{a)} Un data frame que represente a la cuenta. En la misma se debe
detallar lo que debe pagar cada cliente.
\end{quote}

\begin{quote}
\emph{b)} Una variable en donde se aloje el total de la cuenta a pagar
por la mesa.
\end{quote}

\begin{quote}
\textbf{Nota}: en caso de leer los archivos mencionados con la función
read.table(), introduzca como argumento \textbf{stringsAsFactors =
FALSE}. De lo contrario se leerán las columnas de la tabla como
factores.
\end{quote}

\end{document}
